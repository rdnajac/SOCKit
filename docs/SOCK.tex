% SOCKit language definition for listings
\lstdefinelanguage{SOCKit}{
    % You can specify the language to base on, for example, 'Python' or 'C'.
    % If left empty, it doesn't inherit any styles from other languages.
    % language=Python,
    
    % List of keywords in the SOCKit language
    keywords={true, false, bool},
    
    % Comment styles in the SOCKit language
    % Define line comment style (if applicable):
    % morecomment=[l]{//},
    % Define block comment style (if applicable):
    % morecomment=[s]{/*}{*/},
    
    % String styles in the SOCKit language
    % Define single-quoted strings (if applicable):
    % morestring=[b]',
    % Define double-quoted strings (if applicable):
    % morestring=[b]",
    
    % The style for keywords, comments, and strings can be customized here
    % keywordstyle=\color{blue},
    % commentstyle=\color{green},
    % stringstyle=\color{red},
    
    % Basic styling for code listing
    basicstyle=\ttfamily\footnotesize,
    breakatwhitespace=false,
    breaklines=true,
    captionpos=b,
    keepspaces=true,
    numbers=left,                    % Line numbers on the left
    numberstyle=\tiny,               % Line numbers styling
    showspaces=false,
    showstringspaces=false,
    showtabs=false,
    tabsize=2,
    
    % If you want to add a background color to listings, uncomment the following line:
    % backgroundcolor=\color{backcolour},
}

% End of SOCKit language definition

% To use this language definition in another .tex file, you would include the following lines in the preamble:
%
% \usepackage{listings}
% % SOCKit language definition for listings
\lstdefinelanguage{SOCKit}{
    % You can specify the language to base on, for example, 'Python' or 'C'.
    % If left empty, it doesn't inherit any styles from other languages.
    % language=Python,
    
    % List of keywords in the SOCKit language
    keywords={true, false, bool},
    
    % Comment styles in the SOCKit language
    % Define line comment style (if applicable):
    % morecomment=[l]{//},
    % Define block comment style (if applicable):
    % morecomment=[s]{/*}{*/},
    
    % String styles in the SOCKit language
    % Define single-quoted strings (if applicable):
    % morestring=[b]',
    % Define double-quoted strings (if applicable):
    % morestring=[b]",
    
    % The style for keywords, comments, and strings can be customized here
    % keywordstyle=\color{blue},
    % commentstyle=\color{green},
    % stringstyle=\color{red},
    
    % Basic styling for code listing
    basicstyle=\ttfamily\footnotesize,
    breakatwhitespace=false,
    breaklines=true,
    captionpos=b,
    keepspaces=true,
    numbers=left,                    % Line numbers on the left
    numberstyle=\tiny,               % Line numbers styling
    showspaces=false,
    showstringspaces=false,
    showtabs=false,
    tabsize=2,
    
    % If you want to add a background color to listings, uncomment the following line:
    % backgroundcolor=\color{backcolour},
}

% End of SOCKit language definition

% To use this language definition in another .tex file, you would include the following lines in the preamble:
%
% \usepackage{listings}
% % SOCKit language definition for listings
\lstdefinelanguage{SOCKit}{
    % You can specify the language to base on, for example, 'Python' or 'C'.
    % If left empty, it doesn't inherit any styles from other languages.
    % language=Python,
    
    % List of keywords in the SOCKit language
    keywords={true, false, bool},
    
    % Comment styles in the SOCKit language
    % Define line comment style (if applicable):
    % morecomment=[l]{//},
    % Define block comment style (if applicable):
    % morecomment=[s]{/*}{*/},
    
    % String styles in the SOCKit language
    % Define single-quoted strings (if applicable):
    % morestring=[b]',
    % Define double-quoted strings (if applicable):
    % morestring=[b]",
    
    % The style for keywords, comments, and strings can be customized here
    % keywordstyle=\color{blue},
    % commentstyle=\color{green},
    % stringstyle=\color{red},
    
    % Basic styling for code listing
    basicstyle=\ttfamily\footnotesize,
    breakatwhitespace=false,
    breaklines=true,
    captionpos=b,
    keepspaces=true,
    numbers=left,                    % Line numbers on the left
    numberstyle=\tiny,               % Line numbers styling
    showspaces=false,
    showstringspaces=false,
    showtabs=false,
    tabsize=2,
    
    % If you want to add a background color to listings, uncomment the following line:
    % backgroundcolor=\color{backcolour},
}

% End of SOCKit language definition

% To use this language definition in another .tex file, you would include the following lines in the preamble:
%
% \usepackage{listings}
% % SOCKit language definition for listings
\lstdefinelanguage{SOCKit}{
    % You can specify the language to base on, for example, 'Python' or 'C'.
    % If left empty, it doesn't inherit any styles from other languages.
    % language=Python,
    
    % List of keywords in the SOCKit language
    keywords={true, false, bool},
    
    % Comment styles in the SOCKit language
    % Define line comment style (if applicable):
    % morecomment=[l]{//},
    % Define block comment style (if applicable):
    % morecomment=[s]{/*}{*/},
    
    % String styles in the SOCKit language
    % Define single-quoted strings (if applicable):
    % morestring=[b]',
    % Define double-quoted strings (if applicable):
    % morestring=[b]",
    
    % The style for keywords, comments, and strings can be customized here
    % keywordstyle=\color{blue},
    % commentstyle=\color{green},
    % stringstyle=\color{red},
    
    % Basic styling for code listing
    basicstyle=\ttfamily\footnotesize,
    breakatwhitespace=false,
    breaklines=true,
    captionpos=b,
    keepspaces=true,
    numbers=left,                    % Line numbers on the left
    numberstyle=\tiny,               % Line numbers styling
    showspaces=false,
    showstringspaces=false,
    showtabs=false,
    tabsize=2,
    
    % If you want to add a background color to listings, uncomment the following line:
    % backgroundcolor=\color{backcolour},
}

% End of SOCKit language definition

% To use this language definition in another .tex file, you would include the following lines in the preamble:
%
% \usepackage{listings}
% \input{SOCK.tex}
%
% And then use the lstlisting environment with the language=SOCKit option in your document:
%
% \begin{lstlisting}[language=SOCKit]
% # Your SOCKit code here
% \end{lstlisting}
%
% Ensure that SOCK.tex is in the same directory as your main .tex file, or adjust the path in the \input{} command accordingly.


%
% And then use the lstlisting environment with the language=SOCKit option in your document:
%
% \begin{lstlisting}[language=SOCKit]
% # Your SOCKit code here
% \end{lstlisting}
%
% Ensure that SOCK.tex is in the same directory as your main .tex file, or adjust the path in the \input{} command accordingly.


%
% And then use the lstlisting environment with the language=SOCKit option in your document:
%
% \begin{lstlisting}[language=SOCKit]
% # Your SOCKit code here
% \end{lstlisting}
%
% Ensure that SOCK.tex is in the same directory as your main .tex file, or adjust the path in the \input{} command accordingly.


%
% And then use the lstlisting environment with the language=SOCKit option in your document:
%
% \begin{lstlisting}[language=SOCKit]
% # Your SOCKit code here
% \end{lstlisting}
%
% Ensure that SOCK.tex is in the same directory as your main .tex file, or adjust the path in the \input{} command accordingly.

