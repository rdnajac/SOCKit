%--------------------
% Packages
%-------------------
\documentclass[12pt,a4paper]{article}
\usepackage[utf8x]{inputenc}
\usepackage[T1]{fontenc}
\usepackage{mathptmx} % Use Times Font

\usepackage[pdftex]{graphicx} % Required for including pictures
\usepackage[pdftex,linkcolor=black,pdfborder={0 0 0}]{hyperref} % Format links for pdf
\usepackage{calc} % To reset the counter in the document after title page
\usepackage{enumitem} % Includes lists

\frenchspacing % No double spacing between sentences
\linespread{1.2} % Set linespace
\usepackage[a4paper, lmargin=0.1666\paperwidth, rmargin=0.1666\paperwidth, tmargin=0.1111\paperheight, bmargin=0.1111\paperheight]{geometry}

\usepackage[all]{nowidow} % Tries to remove widows
\usepackage[protrusion=true,expansion=true]{microtype} % Improves typography, load after fontpackage is selected

\usepackage{listings} % For code listing

\hypersetup{
    pdftitle={SOCKit_proposal},
    pdfauthor={rdn2108},
    pdfsubject={coms4115}
}

%-----------------------
% Code block formatting
%-----------------------
% Custom listing style
\lstdefinestyle{mystyle}{
    basicstyle=\ttfamily\footnotesize,
    breakatwhitespace=false,
    breaklines=true,
    captionpos=b,
    keepspaces=true,
    numbers=left,
    numbersep=5pt,
    showspaces=false,
    showstringspaces=false,
    showtabs=false,
    tabsize=2
}

\lstset{style=mystyle}

%-----------------------
% Maketitle
%-----------------------
\title{SOCKit: Socket-Oriented Concurrency Kit}
\author{
    Quentin Autry\\
    \texttt{(System Architect)}
    \and
    Quinn Booth\\
    \texttt{(Language Guru)}
    \and
    Hakim El Ghazi\\
    \texttt{(Float Goat)}
    \and
    Ryan Najac\\
    \texttt{(Manager)}
    \and
    Stacey Yao\\
    \texttt{(Tester)}
}

%-----------------------
% Begin document
%-----------------------
\begin{document}
\maketitle

\section{Motivation}
The Socket-Oriented Concurrency Kit Unix combines Unix socket programming with bash-like syntax for safe and easy concurrency with the robustness of a C program. With improvements like warning compiler warnings for potential race conditions and deadlocks, SOCKit aims to make concurrent programming safer and more accessible leveraging `source` and `destination` data type alongside a pipelining syntax. The language facilitates (pseudo-)multi-threaded and asynchronous ininter-client communication and simplifies the flow of data across devices. SOCKit is an ideal choice for developers focused on efficient data transfer and real-time communication.

\section{Language Paradigms and Features}

SOCKit is an imperative, weakly but statically typed language with static scope and a strict evaluation order, designed for clarity and predictability. It supports efficient concurrency, socket programming, and data pipelining to enhance processing efficiency. Future considerations include compiler-enforced mutual exclusion for concurrency safety, built-in regex support for easier text processing, and automatic thread garbage collection for improved memory management.

\section{“Hello World”}
\begin{lstlisting}[numbers=none]
send{"Hello, world!"}-> DEST & DEST <- 192.168.1.1:8080
\end{lstlisting}


\newpage
\section{“Language in One Slide”}
\begin{lstlisting}
SRC <- localhost
DEST1 <- 192.168.1.1:10
DEST2 <- 192.168.1.2:20
DEST3 <- 192.168.1.3


func encrypt (DATA, key) -> encrypted_msg
-> ~DATA ? "EMPTY MSG! "
-> DATA ? msg == key ? msg

set encrypted_msg ""
for c in i in msg
set encrypted_char (c + key[i % len(key)]) % 256
encrypted_msg += encrypted_char


set protocol send_by_character (msg) -{}->
for i in msg
-> i


set protocol wait_to_send (port, msg) \
-{send_by_character}->
set received listen(SRC:port)
wait while ~receive
send{"BAD"} ? ~received : -> msg


!! Send encrypted msg to DEST1, then DEST2 and port 30 of DEST3
send{"Hello", "aX82kLei19g"} -> encrypt -> \
DEST1 | DEST2 & DEST3:30

\end{lstlisting}
\end{document}

