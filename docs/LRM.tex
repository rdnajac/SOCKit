\documentclass[12pt,a4paper]{article} % preamble {{{1
\usepackage[a4paper, margin=1in]{geometry}
\usepackage{microtype}      % improved text appearance
\usepackage[T1]{fontenc}    % better font encoding
\usepackage[utf8]{inputenc} % input encoding
\usepackage{mathptmx}       % times new roman
\usepackage{listings}       % code listings
\usepackage{xcolor}         % color definitions
\usepackage{enumitem}       % custom list configurations
\usepackage{tocloft}        % table of contents customization

\renewcommand{\baselinestretch}{1.2}

% import language definition
%% SOCKit language definition for listings
\lstdefinelanguage{SOCKit}{
    % You can specify the language to base on, for example, 'Python' or 'C'.
    % If left empty, it doesn't inherit any styles from other languages.
    % language=Python,
    
    % List of keywords in the SOCKit language
    keywords={true, false, bool},
    
    % Comment styles in the SOCKit language
    % Define line comment style (if applicable):
    % morecomment=[l]{//},
    % Define block comment style (if applicable):
    % morecomment=[s]{/*}{*/},
    
    % String styles in the SOCKit language
    % Define single-quoted strings (if applicable):
    % morestring=[b]',
    % Define double-quoted strings (if applicable):
    % morestring=[b]",
    
    % The style for keywords, comments, and strings can be customized here
    % keywordstyle=\color{blue},
    % commentstyle=\color{green},
    % stringstyle=\color{red},
    
    % Basic styling for code listing
    basicstyle=\ttfamily\footnotesize,
    breakatwhitespace=false,
    breaklines=true,
    captionpos=b,
    keepspaces=true,
    numbers=left,                    % Line numbers on the left
    numberstyle=\tiny,               % Line numbers styling
    showspaces=false,
    showstringspaces=false,
    showtabs=false,
    tabsize=2,
    
    % If you want to add a background color to listings, uncomment the following line:
    % backgroundcolor=\color{backcolour},
}

% End of SOCKit language definition

% To use this language definition in another .tex file, you would include the following lines in the preamble:
%
% \usepackage{listings}
% % SOCKit language definition for listings
\lstdefinelanguage{SOCKit}{
    % You can specify the language to base on, for example, 'Python' or 'C'.
    % If left empty, it doesn't inherit any styles from other languages.
    % language=Python,
    
    % List of keywords in the SOCKit language
    keywords={true, false, bool},
    
    % Comment styles in the SOCKit language
    % Define line comment style (if applicable):
    % morecomment=[l]{//},
    % Define block comment style (if applicable):
    % morecomment=[s]{/*}{*/},
    
    % String styles in the SOCKit language
    % Define single-quoted strings (if applicable):
    % morestring=[b]',
    % Define double-quoted strings (if applicable):
    % morestring=[b]",
    
    % The style for keywords, comments, and strings can be customized here
    % keywordstyle=\color{blue},
    % commentstyle=\color{green},
    % stringstyle=\color{red},
    
    % Basic styling for code listing
    basicstyle=\ttfamily\footnotesize,
    breakatwhitespace=false,
    breaklines=true,
    captionpos=b,
    keepspaces=true,
    numbers=left,                    % Line numbers on the left
    numberstyle=\tiny,               % Line numbers styling
    showspaces=false,
    showstringspaces=false,
    showtabs=false,
    tabsize=2,
    
    % If you want to add a background color to listings, uncomment the following line:
    % backgroundcolor=\color{backcolour},
}

% End of SOCKit language definition

% To use this language definition in another .tex file, you would include the following lines in the preamble:
%
% \usepackage{listings}
% % SOCKit language definition for listings
\lstdefinelanguage{SOCKit}{
    % You can specify the language to base on, for example, 'Python' or 'C'.
    % If left empty, it doesn't inherit any styles from other languages.
    % language=Python,
    
    % List of keywords in the SOCKit language
    keywords={true, false, bool},
    
    % Comment styles in the SOCKit language
    % Define line comment style (if applicable):
    % morecomment=[l]{//},
    % Define block comment style (if applicable):
    % morecomment=[s]{/*}{*/},
    
    % String styles in the SOCKit language
    % Define single-quoted strings (if applicable):
    % morestring=[b]',
    % Define double-quoted strings (if applicable):
    % morestring=[b]",
    
    % The style for keywords, comments, and strings can be customized here
    % keywordstyle=\color{blue},
    % commentstyle=\color{green},
    % stringstyle=\color{red},
    
    % Basic styling for code listing
    basicstyle=\ttfamily\footnotesize,
    breakatwhitespace=false,
    breaklines=true,
    captionpos=b,
    keepspaces=true,
    numbers=left,                    % Line numbers on the left
    numberstyle=\tiny,               % Line numbers styling
    showspaces=false,
    showstringspaces=false,
    showtabs=false,
    tabsize=2,
    
    % If you want to add a background color to listings, uncomment the following line:
    % backgroundcolor=\color{backcolour},
}

% End of SOCKit language definition

% To use this language definition in another .tex file, you would include the following lines in the preamble:
%
% \usepackage{listings}
% \input{SOCK.tex}
%
% And then use the lstlisting environment with the language=SOCKit option in your document:
%
% \begin{lstlisting}[language=SOCKit]
% # Your SOCKit code here
% \end{lstlisting}
%
% Ensure that SOCK.tex is in the same directory as your main .tex file, or adjust the path in the \input{} command accordingly.


%
% And then use the lstlisting environment with the language=SOCKit option in your document:
%
% \begin{lstlisting}[language=SOCKit]
% # Your SOCKit code here
% \end{lstlisting}
%
% Ensure that SOCK.tex is in the same directory as your main .tex file, or adjust the path in the \input{} command accordingly.


%
% And then use the lstlisting environment with the language=SOCKit option in your document:
%
% \begin{lstlisting}[language=SOCKit]
% # Your SOCKit code here
% \end{lstlisting}
%
% Ensure that SOCK.tex is in the same directory as your main .tex file, or adjust the path in the \input{} command accordingly.



\title{SOCKit Reference Manual}
\author{
    \begin{tabular}{rl} % "r" for right-aligned names, "l" for left-aligned titles
    & \\
    & \\
    & \\
    & \\
    & \\
    & \\
    & \\
    & \\
    & \\
    Quentin Autry & \texttt{System Architect} \\
    Hakim El Ghazi & \texttt{Float Goat} \\
    Ryan Najac & \texttt{Manager} \\
    Misha Smirnov & \texttt{Language Guru} \\
    Stacey Yao & \texttt{Tester} \\
    & \\
    & \\
    & \\
    & \\
    & \\
    & \\
    & \\
    & \\
    & \\
    \end{tabular}
}
\date{Columbia University\\COMS W4115\\Programming Languages and Translators\\March 22, 2024}

\begin{document}
\maketitle
\thispagestyle{empty}
\newpage
\tableofcontents
\newpage
% 1}}}
\section{Overview}
Sockit integrates the structured syntax of C with scripting conveniences, emphasizing Unix-like operations, concurrency, and streamlined data handling. It is optimized for system-level tasks, offering an efficient pathway for both compiled and script-like programming paradigms.

\section{Lexical Conventions}
The lexical components of Sockit include identifiers, keywords, constants, string literals, various operators, and separators. Spaces, tabs, and newlines serve to delineate tokens, except when their presence is integral to token formation.

\subsection{Comments}
Sockit supports block (\texttt{/* ... */}) and line (\texttt{//}) comments, akin to C. Unlike C, Sockit allows comments to nest, facilitating more flexible code annotation.

\begin{verbatim}
/* Block comment example */
// Line comment example
/* A comment /* nested within */ another */
\end{verbatim}

\subsection{Identifiers}
Identifiers begin with a letter or underscore, followed by any combination of letters, digits, and underscores. Sockit treats uppercase and lowercase letters as distinct.

\begin{verbatim}
int identifier1;
data _dataIdentifier;
\end{verbatim}

\section{Keywords and Constants}
Reserved keywords in Sockit are used to denote control structures, data types, and other language constructs. Sockit enhances type simplicity and introduces scripting-like conveniences.

Sockit supports integral constants in various bases: decimal, hexadecimal (prefixed by \texttt{0x}), and binary (prefixed by \texttt{0b}).

\begin{verbatim}
int dec = 10;
int hex = 0x1A;
int bin = 0b1010;
\end{verbatim}

\subsection{Special Keywords}
\texttt{data} serves as a generic pointer type, akin to void pointers but with an emphasis on data-agnostic operations. \texttt{null} represents the absence of data.

\begin{verbatim}
data genericData = null;
\end{verbatim}

\section{Concurrency and Atomic Operations}
\subsection{Threads}
Sockit introduces constructs for managing concurrent execution flows, simplifying the use of POSIX threads.

\begin{verbatim}
thread t = thread_create(function, arg);
thread_join(t);
\end{verbatim}

\subsection{Atomics}
Atomic variables and operations ensure data integrity across concurrent executions.

\begin{verbatim}
@int atomicCounter = 0;
atomic_increment(&atomicCounter);
\end{verbatim}

\textbf{Example:}
\begin{verbatim}
statement ::= if_statement | while_statement | for_statement;
if_statement ::= 'if' '(' expression ')' statement ['else' statement];
\end{verbatim}
\section{Syntax Summary}
The syntax of Sockit is designed to support both the structured programming model of C and the dynamic, scripting capabilities familiar to Unix shell users. This summary provides an overview of the syntactic elements that constitute the Sockit language.

\subsection{Basic Syntax}
\begin{itemize}
    \item \texttt{\textbf{statements}} include declarations, assignments, control flow statements (\texttt{\textbf{if}}, \texttt{\textbf{for}}, \texttt{\textbf{while}}, \texttt{\textbf{switch}}), and function calls.
    \item \texttt{\textbf{expression}} encompasses operations among variables, literals, and function calls that return values. Expressions are used for calculation, logic operations, and to determine flow control.
\end{itemize}

\subsection{Function Declaration and Invocation}
\begin{verbatim}
return_type function_name(parameters) {
    // function body
}

function_name(arguments);
\end{verbatim}

\subsection{Control Structures}
\begin{itemize}
    \item \texttt{\textbf{if (condition) {}}}: Executes a block of statements if the condition is true.
    \item \texttt{\textbf{for (initialization; condition; increment) {}}}: Executes a loop that continues as long as the condition evaluates to true.
    \item \texttt{\textbf{while (condition) {}}}: Continues to execute a block of statements as long as the condition remains true.
\end{itemize}

\subsection{Concurrent Execution and Atomics}
\begin{verbatim}
thread t = thread_create(function, arg);
thread_join(t);

@int atomicVar;
atomic_increment(&atomicVar);
\end{verbatim}

\subsection{Data Types and Variables}
\begin{verbatim}
int main() {
    int integerVar = 10;
    data ptr = null;
    @int atomicInt = 0;
}
\end{verbatim}

\appendix
\section*{APPENDIX}
\addcontentsline{toc}{section}{APPENDIX: Additional Features and Extensions} % Adds the appendix to the table of contents

\subsection*{Unix-like Scripting Features}
Sockit's syntax extends traditional C by incorporating Unix-like scripting features for more dynamic programming:
\begin{verbatim}
command1 | command2; // Pipes the output of command1 to command2
data result = `command`; // Executes a shell command and captures its output
\end{verbatim}

\subsection*{Background Process Execution}
\begin{verbatim}
run_background_process(command);
\end{verbatim}
Enables execution of long-running or daemon processes in the background, without blocking the main execution thread.

\subsection*{Advanced Atomic Operations}
\begin{verbatim}
@int atomicVar;
atomic_compare_exchange(&atomicVar, expected, desired);
\end{verbatim}
Provides mechanisms for compare-and-swap operations, crucial for lock-free concurrent algorithms.

This appendix highlights Sockit's unique features, designed to bridge the gap between structured and script-based programming within system-level applications.

\subsection*{Syntax Summary}
\subsubsection*{Expressions and Operations}
\begin{verbatim}
expression:
    primary
    * expression                   // multiplication
    / expression                   // division
    + expression                   // addition
    - expression                   // subtraction
    expression ? expression : expression // conditional
    expression == expression       // equality
    expression != expression       // inequality
    expression >= expression       // greater than or equal
    expression <= expression       // less than or equal
    expression > expression        // greater than
    expression < expression        // less than
    expression && expression       // logical AND
    expression || expression       // logical OR
    !expression                    // logical NOT
    expression & expression        // bitwise AND
    expression | expression        // bitwise OR
    expression ^ expression        // bitwise XOR
    ~expression                    // bitwise NOT
    expression << expression       // left shift
    expression >> expression       // right shift

\end{verbatim}

\subsubsection*{Data Types and Declarations}
\begin{verbatim}
types:
    int
    float
    bool
    char
    data // Special Sockit type for generic data handling

declarations:
    type specifier [list of variables];
    data variable_name; // Generic data variable declaration

\end{verbatim}

\subsubsection*{Unix-like Features}
\begin{verbatim}
// Data pipelining
command1 | command2; // Pipe output of command1 to command2

// Redirection
command > file;    // Redirect standard output to file
command < file;    // Redirect file to standard input of command
command >> file;   // Append standard output to file

// Background process
command &;         // Execute command in background

\end{verbatim}

\subsubsection*{Concurrency and Atomics}
\begin{verbatim}
// Atomic operations (indicated by @ prefix)
@int atomic_var;                 // Declaration of atomic integer
@atomic_var++;                   // Atomic increment
@atomic_var--;                   // Atomic decrement
@atomic_compare_exchange(&atomic_var, expected, desired);

\end{verbatim}

\subsubsection*{Control Structures}
\begin{verbatim}
/* Standard C control structures are supported,
 * including if-else, while, switch-case
 */

if (expression) {
    // branch 1
} else {
    // branch 2
}

while (condition) {
    // loop body
}

switch (expression) {
    case value1:
        // case 1 actions
        break;
    case value2:
        // case 2 actions
        break;
    default:
        // default actions
}

\end{verbatim}
\end{document}
