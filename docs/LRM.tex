\documentclass[12pt,a4paper]{article}
\usepackage[a4paper, margin=1in]{geometry}
\usepackage{microtype}      % improved text appearance
\usepackage[T1]{fontenc}    % better font encoding
\usepackage[utf8]{inputenc} % input encoding
\usepackage{mathptmx}       % times new roman
\usepackage{listings}       % code listings
\usepackage{xcolor}         % color definitions
\usepackage{enumitem}       % custom list configurations
\usepackage{tocloft}        % table of contents customization

\renewcommand{\baselinestretch}{1.2}

% import language definition
%% SOCKit language definition for listings
\lstdefinelanguage{SOCKit}{
    % You can specify the language to base on, for example, 'Python' or 'C'.
    % If left empty, it doesn't inherit any styles from other languages.
    % language=Python,
    
    % List of keywords in the SOCKit language
    keywords={true, false, bool},
    
    % Comment styles in the SOCKit language
    % Define line comment style (if applicable):
    % morecomment=[l]{//},
    % Define block comment style (if applicable):
    % morecomment=[s]{/*}{*/},
    
    % String styles in the SOCKit language
    % Define single-quoted strings (if applicable):
    % morestring=[b]',
    % Define double-quoted strings (if applicable):
    % morestring=[b]",
    
    % The style for keywords, comments, and strings can be customized here
    % keywordstyle=\color{blue},
    % commentstyle=\color{green},
    % stringstyle=\color{red},
    
    % Basic styling for code listing
    basicstyle=\ttfamily\footnotesize,
    breakatwhitespace=false,
    breaklines=true,
    captionpos=b,
    keepspaces=true,
    numbers=left,                    % Line numbers on the left
    numberstyle=\tiny,               % Line numbers styling
    showspaces=false,
    showstringspaces=false,
    showtabs=false,
    tabsize=2,
    
    % If you want to add a background color to listings, uncomment the following line:
    % backgroundcolor=\color{backcolour},
}

% End of SOCKit language definition

% To use this language definition in another .tex file, you would include the following lines in the preamble:
%
% \usepackage{listings}
% % SOCKit language definition for listings
\lstdefinelanguage{SOCKit}{
    % You can specify the language to base on, for example, 'Python' or 'C'.
    % If left empty, it doesn't inherit any styles from other languages.
    % language=Python,
    
    % List of keywords in the SOCKit language
    keywords={true, false, bool},
    
    % Comment styles in the SOCKit language
    % Define line comment style (if applicable):
    % morecomment=[l]{//},
    % Define block comment style (if applicable):
    % morecomment=[s]{/*}{*/},
    
    % String styles in the SOCKit language
    % Define single-quoted strings (if applicable):
    % morestring=[b]',
    % Define double-quoted strings (if applicable):
    % morestring=[b]",
    
    % The style for keywords, comments, and strings can be customized here
    % keywordstyle=\color{blue},
    % commentstyle=\color{green},
    % stringstyle=\color{red},
    
    % Basic styling for code listing
    basicstyle=\ttfamily\footnotesize,
    breakatwhitespace=false,
    breaklines=true,
    captionpos=b,
    keepspaces=true,
    numbers=left,                    % Line numbers on the left
    numberstyle=\tiny,               % Line numbers styling
    showspaces=false,
    showstringspaces=false,
    showtabs=false,
    tabsize=2,
    
    % If you want to add a background color to listings, uncomment the following line:
    % backgroundcolor=\color{backcolour},
}

% End of SOCKit language definition

% To use this language definition in another .tex file, you would include the following lines in the preamble:
%
% \usepackage{listings}
% % SOCKit language definition for listings
\lstdefinelanguage{SOCKit}{
    % You can specify the language to base on, for example, 'Python' or 'C'.
    % If left empty, it doesn't inherit any styles from other languages.
    % language=Python,
    
    % List of keywords in the SOCKit language
    keywords={true, false, bool},
    
    % Comment styles in the SOCKit language
    % Define line comment style (if applicable):
    % morecomment=[l]{//},
    % Define block comment style (if applicable):
    % morecomment=[s]{/*}{*/},
    
    % String styles in the SOCKit language
    % Define single-quoted strings (if applicable):
    % morestring=[b]',
    % Define double-quoted strings (if applicable):
    % morestring=[b]",
    
    % The style for keywords, comments, and strings can be customized here
    % keywordstyle=\color{blue},
    % commentstyle=\color{green},
    % stringstyle=\color{red},
    
    % Basic styling for code listing
    basicstyle=\ttfamily\footnotesize,
    breakatwhitespace=false,
    breaklines=true,
    captionpos=b,
    keepspaces=true,
    numbers=left,                    % Line numbers on the left
    numberstyle=\tiny,               % Line numbers styling
    showspaces=false,
    showstringspaces=false,
    showtabs=false,
    tabsize=2,
    
    % If you want to add a background color to listings, uncomment the following line:
    % backgroundcolor=\color{backcolour},
}

% End of SOCKit language definition

% To use this language definition in another .tex file, you would include the following lines in the preamble:
%
% \usepackage{listings}
% \input{SOCK.tex}
%
% And then use the lstlisting environment with the language=SOCKit option in your document:
%
% \begin{lstlisting}[language=SOCKit]
% # Your SOCKit code here
% \end{lstlisting}
%
% Ensure that SOCK.tex is in the same directory as your main .tex file, or adjust the path in the \input{} command accordingly.


%
% And then use the lstlisting environment with the language=SOCKit option in your document:
%
% \begin{lstlisting}[language=SOCKit]
% # Your SOCKit code here
% \end{lstlisting}
%
% Ensure that SOCK.tex is in the same directory as your main .tex file, or adjust the path in the \input{} command accordingly.


%
% And then use the lstlisting environment with the language=SOCKit option in your document:
%
% \begin{lstlisting}[language=SOCKit]
% # Your SOCKit code here
% \end{lstlisting}
%
% Ensure that SOCK.tex is in the same directory as your main .tex file, or adjust the path in the \input{} command accordingly.



% metadata for title page
\title{SOCKit Reference Manual}
\author{
    \begin{tabular}{rl} % "r" for right-aligned names, "l" for left-aligned titles
    Quentin Autry & \texttt{System Architect} \\
    Hakim El Ghazi & \texttt{Float Goat} \\
    Ryan Najac & \texttt{Manager} \\
    Misha Smirnov & \texttt{Language Guru} \\
    Stacey Yao & \texttt{Tester}
    \end{tabular}
}
\date{COMS W4115 Programming Languages and Translators\\March 22, 2024}

\begin{document}
\thispagestyle{empty}
\maketitle
\newpage
\tableofcontents % Generates the table of contents
\newpage

\section{Overview}
Sections and subsections will be automatically added to the table of contents.

\section{Lexical conventions}
There are six kinds of tokens: identifiers, keywords, constants, strings, expression operators, and other separators.

In general, blanks, tabs, newlines, and comments as described below are ignored except as they serve to separate tokens. At least one of these characters is required to separate otherwise adjacent identifiers, constants, and certain operator-pairs.

\subsection{Comments}
The characters `/*` introduce a comment, which terminates with the characters `*/`. Single-line comments prefixed with `//` are also supported.

\subsection{Identifiers (Names)}
An identifier is a sequence of letters and digits; the first character must be a letter. The underscore character \texttt{\_} is considered a letter. Upper and lower case letters are distinct.

\section{Keywords}
The following identifiers are reserved for use as keywords and may not be used otherwise:

\subsection{Constants}
There are several kinds of constants, as follows:
\subsubsection{Integer Constants}
In keeping with the spirit of minimalism, we only support nonnegative integer constants. The value of an integer constant is the sequence of digits. We do, however, support different bases for integer constants. The base is indicated by a prefix that is either `0x` or `0X` for hexadecimal, `0b` for binary, and nothing for decimal.

\subsubsection{data}
The keyword \texttt{data} is used to declare a variable. It works similar to how Unix treats everything as files (i.e., everything data). It's basically a void pointer, letting the programmer know that this data exists somewhere.

\subsubsection{null}
The keyword \texttt{null} is used to declare a null value. This is the absence of a value. Uninitialized variables are implicitly null.

\subsection{Primitive Types}
SOCKit has primitive types: \texttt{int}, \texttt{float}, \texttt{bool}, and \texttt{char}.

\section{Syntax Notation}
The syntax notation used in this manual follows the syntax notation in the "C Reference Manual" (Ritchie, 1975):

\begin{itemize}
    \item \texttt{italic} is used for non-terminal symbols.
    \item \texttt{bold} is used for terminal symbols.
    \item \texttt{[ ... ]} is used for optional items.
    \item \texttt{\{ ... \}} is used for repeated items.
    \item \texttt{|} is used to separate alternatives.
    \item \texttt{...} is used for a syntactic category that can be repeated.
\end{itemize}

\section{Data Types}
No data types, everything is void pointers.

\section{Expressions}
The precedence of expression operators is the same as the order of the major subsections of this section (highest precedence first).

\appendix
\section*{APPENDIX}
\addcontentsline{toc}{section}{APPENDIX: Syntax Summary} % Adds the appendix to the table of contents
\subsection*{Syntax Summary}

\begin{enumerate}
    \item Expressions.
    \begin{verbatim}
    expression:
        primary
        * expression
    \end{verbatim}
    \item Declarations...
    % Add your declarations syntax here
\end{enumerate}

\end{document}
